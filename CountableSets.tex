\documentclass{article}
\usepackage{amsmath, amssymb}
\usepackage{setspace}
\onehalfspacing

\begin{document}

\section*{Question 7.1}
Two sets have the same size if there is a bijection between them. Prove that there are infinitely many different sizes of infinite sets; that is, that there are at least a countable number of infinite sets, no two of which have the same size.

\textbf{Answer:} Cantor's Diagonal Argument shows that there are uncountably many sizes of infinite sets. This is because the power set of the natural numbers, denoted as $P(\mathbb{N})$, is uncountable. There is no bijection between $\mathbb{N}$ and $P(\mathbb{N})$. Therefore, there are infinitely many different sizes of infinite sets, and they are uncountable.

\section*{Question 7.3}
Johnny is skeptical about the proof of Theorem 7.4. He agrees that the diagonalization process yields a set of natural numbers $D$ that is not in the original list $S_0, S_1, \ldots$. He claims that the new set can be accommodated by moving all the indices up by 1 and sliding the new set in at the beginning. What is wrong with his argument?

\textbf{Answer:} Johnny's argument is flawed because the construction of $D$ ensures that it cannot be equal to any $S_i$ for any $i$. If we were to move all the indices up by 1 and set $T_0 = D$ and $T_{i+1} = S_i$ for each $i$, we would simply have a new list of sets, but $D$ would still not be in this list. The reason is that $D$ differs from each $S_i$ at the position where the diagonal and the $i$-th row intersect, making it impossible to accommodate $D$ by shifting the indices. Therefore, Johnny's argument is invalid.

\section*{Question 7.5}
Which of the following scenarios are possible? Provide examples or explanations.

(a) The set difference of two uncountable sets is countable.

(b) The set difference of two countably infinite sets is countably infinite.

(c) The power set of a countable set is countable.

(d) The union of a collection of finite sets is countably infinite.

(e) The union of a collection of finite sets is uncountable.

(f) The intersection of two uncountable sets is empty.

\textbf{Answers:}

(a) The set difference of two uncountable sets can be countable. For example, consider the sets $A = (0, 1)$ and $B = (0, 0.5)$. The set difference $A \setminus B = (0.5, 1)$ is a countable interval.

(b) The set difference of two countably infinite sets can be countably infinite. For instance, consider the sets of even natural numbers $A = \{2, 4, 6, \ldots\}$ and $B = \{4, 8, 12, \ldots\}$. The set difference $A \setminus B = \{2, 6, 10, \ldots\}$ is countably infinite.

(c) The power set of a countable set is countable. For example, if we have a countable set $\mathbb{N}$, then its power set $P(\mathbb{N})$ is countable.

(d) The union of a collection of finite sets is countably infinite. Finite sets contain a finite number of elements, and the union of countably many finite sets is still countably infinite.

(e) The union of a collection of finite sets cannot be uncountable. It remains countably infinite or becomes finite.

(f) The intersection of two uncountable sets can be empty. For example, the intersection of the sets of real numbers and irrational numbers is empty.

\section*{Question 7.7}
(a) Show that there are as many ordered pairs of real numbers between 0 and 1 as there are real numbers in that interval. Exhibit a bijection $f : [0, 1] \times [0, 1] \leftrightarrow [0, 1]$.

(b) Extend the result of part (a) to give a bijection between pairs of nonnegative real numbers and nonnegative real numbers.

\textbf{Answers:}
(a) To show that there are as many ordered pairs of real numbers between 0 and 1 as there are real numbers in that interval, we can exhibit a bijection. Consider a real number in the interval $[0, 1]$ represented as a decimal, such as $0.x_1x_2x_3...$, where each $x_i$ is a digit from 0 to 9. Now, for every real number in $[0, 1]$, pair it with another real number as follows: $0.x_1x_1x_2x_2x_3x_3...$, where each $x_i$ is the corresponding digit from the original number. This creates a bijection between $[0, 1] \times [0, 1]$ and $[0, 1]$.

(b) To extend the result to pairs of nonnegative real numbers, we can apply a similar idea. Given two nonnegative real numbers represented as decimal expansions, pair them by interleaving their digits. For example, if we have $0.x_1x_2x_3...$ and $0.y_1y_2y_3...$, we can create the pair $0.x_1y_1x_2y_2x_3y_3...$. This establishes a bijection between pairs of nonnegative real numbers and nonnegative real numbers.

\section*{Question 7.9}
In each case, state whether the set is finite, countably infinite, or uncountable, and explain why.

(a) The set of all books, where a "book" is a finite sequence of uppercase and lowercase Roman letters, Arabic numerals, the space symbol, and certain punctuation marks: ; , . ' : — ( ) ! ? "

(b) The set of all books of less than 500,000 symbols.

(c) The set of all finite sets of books.

(d) The set of all irrational numbers greater than 0 and less than 1.

(e) The set of all sets of numbers that are divisible by 17.

(f) The set of all sets of even prime numbers.

(g) The set of all sets of powers of 2.

(h) The set of all functions from $\mathbb{Q}$ to $\{0, 1\}$.

\textbf{Answers:}
(a) The set of all books, where a "book" is a finite sequence of characters, is countably infinite. Each book can be represented as a finite string, and the set of all possible finite strings is countably infinite.

(b) The set of all books of less than 500,000 symbols is countably infinite. There are a countable number of books of each possible length, and the union of countably many countable sets is still countable.

(c) The set of all finite sets of books is countably infinite. Each finite set can be enumerated, and the set of all finite sets is a countable union of countable sets.

(d) The set of all irrational numbers greater than 0 and less than 1 is uncountable. This is a well-known result, and it follows from Cantor's Diagonal Argument.

(e) The set of all sets of numbers that are divisible by 17 is uncountable. There are uncountably many subsets of the set of natural numbers.

(f) The set of all sets of even prime numbers is countably infinite. There are only finitely many even prime numbers, so the set of all such sets is countable.

(g) The set of all sets of powers of 2 is uncountable. This is similar to (e), as there are uncountably many subsets of the set of natural numbers.

(h) The set of all functions from $\mathbb{Q}$ to $\{0, 1\}$ is uncountable. This is because there are uncountably many real numbers in $\mathbb{Q}$, and each function maps each real number to either 0 or 1. Therefore, the set of all such functions is uncountable.

\section*{Question 7.11}
Prove Theorem 7.3.

\textbf{Answer:} Theorem 7.3 states that the union of countably many countably infinite sets is countably infinite. To prove this, consider a countable collection of countably infinite sets $S_0, S_1, S_2, \ldots$. Since each of these sets is countably infinite, you can enumerate their elements.

To construct an enumeration of the union of these sets, start by enumerating the elements of $S_0$, then move to the elements of $S_1$, then to $S_2$, and so on. By continuing this process, you will enumerate all the elements in the union, and this enumeration is countable.

\end{document}
